%!TEX root = ../Osteuropaatlas.tex


\section{Sozioökonomische Daten}


\begin{figure}[p]
	\addcontentsline{toc}{subsection}{Lebenserwartung bei Geburt}
	{\centering \maps{Lebenserwartung bei Geburt}}
	\label{map:lebenserw}
	\karte{Lebenserwartung}{2021}{Veränderung 2015 bis 2021}
	\begin{spacing}{1} \scriptsize
		Anm.: Stand 2021\\
		Quelle: Eurostat (2024); Ber. \& Dar. imreg (2024) \end{spacing}
\end{figure}


\begin{figure}[p]
	\addcontentsline{toc}{subsection}{Geburtenrate je Frau}
	{\centering \maps{Geburtenrate je Frau}}
	\label{map:geburten}
	\karte{Geburtenrate}{2020}{Veränderung 2015 bis 2020}
	\begin{spacing}{1} \scriptsize
		Anm.: Geburtenrate in Kinder je Frau; Stand 2020\\
		Quelle: Eurostat (2024); Ber. \& Dar. imreg (2024) \end{spacing}
\end{figure}


\begin{figure}[p]
	\addcontentsline{toc}{subsection}{Armutsgefährdungsrate}
	{\centering \maps{Armutsgefährdungsrate}}
	\label{map:armut}
	\karte{Armutsgefaehrdung}{2022}{Veränderung 2021 bis 2022}
	\begin{spacing}{1} \scriptsize
		Anm.: Anteil Personen mit verfügbarem Einkommen nach Sozialtransfers unterhalb der Armutsrisikogrenze; Armutsrisikogrenze = 60\% des mittleren nationalen verfügbaren Einkommens; Stand 2022\\
		Quelle: Eurostat (2024); Ber. \& Dar. imreg (2024) \end{spacing}
\end{figure}


\begin{figure}[p]
	\addcontentsline{toc}{subsection}{Auspendler}
	{\centering \maps{Auspendler}}
	\label{map:pendler}
	\karte{Auspendler}{2022}{Veränderung 2015 bis 2022}
	\begin{spacing}{1} \scriptsize
		Anm.: Anzahl Beschäftigte mit einer von Wohnregion abweichenden Arbeitsregion; Personen im Alter von 15 bis 64 Jahre; Stand 2022\\
		Quelle: Eurostat (2024); Ber. \& Dar. imreg (2024) \end{spacing}
\end{figure}


\begin{figure}[p]
	\addcontentsline{toc}{subsection}{Weiterbildungsquote}
	{\centering \maps{Weiterbildungsquote}}
	\label{map:weiterbildung}
	\karte{Weiterbildung}{2022}{Veränderung 2015 bis 2022}
	\begin{spacing}{1} \scriptsize
		Anm.: Teilnahme an Bildungs- \& Weiterbildungsmaßnahmen in den letzten vier Wochen zum Umfragezeitpunkt; Personen im Alter von 25 bis 64 Jahren; Stand 2022\\
		Quelle: Eurostat (2024); Ber. \& Dar. imreg (2024) \end{spacing}
\end{figure}


\begin{figure}[p]
	\addcontentsline{toc}{subsection}{Bevölkerungsanteil mit Tertiärabschluss}
	{\centering \maps{Bevölkerungsanteil mit Tertiärabschluss}}
	\label{map:tertbildung}
	\karte{Tertiaere_Bildung}{2022}{Veränderung 2015 bis 2022}
	\begin{spacing}{1} \scriptsize
		Anm.: Tertiärabschluss = Abschluss an Hoch-, Fachschulen, Berufs-, Fachakademien \& Schulen des Gesundheitswesens; Stand 2022\\
		Quelle: Eurostat (2024); Ber. \& Dar. imreg (2024) \end{spacing}
\end{figure}