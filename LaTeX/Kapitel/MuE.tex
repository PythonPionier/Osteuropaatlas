%!TEX root = ../Osteuropaatlas.tex

\section{Metall \& Elektroindustrie}

\begin{table}[!h]
	\addcontentsline{toc}{subsection}{Strukturdaten der M+E-Wirtschaft in Osteuropa}
	\caption{Strukturdaten der M+E-Wirtschaft in Osteuropa}
	\begin{tblr}{
			width = \linewidth,
			rowsep = 4pt,
			colspec = {|X[l,m]|X[r,m]|X[r,m]|X[r,m]|X[r,m]|X[r,m]|},
			row{even} = {bg=imreg!20},
			row{1,2} = {c,bg=imreg,fg=white,font=\bfseries\large, rowsep=4pt},
			row{X} = {font=\bfseries}
			}
		\hline
		\SetCell[r=2]{m} Land & Betriebe (2022) & Beschäftigte (2022) & $\Sigma$ Entgelte (2021) & Umsatz (2022) & {Bruttowert-\\schöpfung (2021)}  \\
		\hline
		& Anzahl & Anzahl & Mio. EUR & Mio. EUR & Mio. EUR \\
		\hline
		Bulgarien & 9.934 & 197.856 & 1.870 & 23.310 & 3.680 \\
		\hline
		Estland & 3.742 & 46.008 & 701 & 7.237 & 1.491 \\
		\hline
		Kroatien & 10.717 & 97.422 & 1.154 & 8.693 & 2.321 \\
		\hline
		Lettland & 3.484 & 34.227 & 1.037 & 3.638 & 768 \\
		\hline
		Litauen & 7.824 & 72.735 & 486 & 7.307 & 2.029 \\
		\hline
		Polen & 120.002 & 1.264.882 & 16.349 & 216.166 & 42.498 \\
		\hline
		Rumänien & 24.972 & 484.698 & 6.620 & 59.839 & 11.960 \\
		\hline
		Slowakei & 50.861 & 302.823 & 4.488 & 71.005 & 11.908 \\
		\hline
		Slowenien & 10.341 & 128.176 & 2.731 & 25.148 & 5.807 \\
		\hline
		Tschechien & 101.034 & 810.191 & 13.054 & 155.711 & 28.127 \\
		\hline
		Ungarn & 30.979 & 441.020 & 6.559 & 89.351 & 15.152 \\
		\hline
		$\Sigma$ Osteuropa & 373.890 & 3.880.038 & 55.049 & 667.405 & 125.743 \\
		\hline
		Sachsen & 1.738 & 190.784 & 8.327 & 57.299 &  \\
		\hline
		Verhältnis Sachsen/ Osteuropa & 0,5\% & 4,9\% & 15,1\% & 8,6\% &  \\
		\hline
	\end{tblr}
	\begin{spacing}{1} \scriptsize
		\vspace{2mm}
		Anm.: WZ 24-30, 32+33; Betriebe ab einem Beschäftigten; Stand 2022\\
		Quelle: Eurostat (2024); Ber. \& Dar. imreg (2024) 
	\end{spacing}
\end{table}

\begin{figure}[p]
	\addcontentsline{toc}{subsection}{M+E-Beschäftigtenzahl}
	{\centering \maps{M+E-Beschäftigtenzahl}}
	\label{map:mebesch}
	\karte{ME_Besch}{2021}{Veränderung 2015 bis 2021}
	\begin{spacing}{1} \scriptsize
		Anm.: Stand 2021\\
		Quelle: Eurostat (2024); Ber. \& Dar. imreg (2024) \end{spacing}
\end{figure}


\begin{figure}[p]
	\addcontentsline{toc}{subsection}{M+E-Unternehmenszahl}
	{\centering \maps{M+E-Unternehmenszahl}}
	\label{map:meunt}
	\karte{ME_Unt}{2021}{Veränderung 2015 bis 2021}
	\begin{spacing}{1} \scriptsize
		Anm.: Stand 2021\\
		Quelle: Eurostat (2024); Ber. \& Dar. imreg (2024) \end{spacing}
\end{figure}


\begin{figure}[p]
	\addcontentsline{toc}{subsection}{M+E-Lohn- \& Gehaltssumme}
	{\centering \maps{M+E-Lohn- \& Gehaltssumme}}
	\label{map:melohn}
	\karte{ME_Entgelt}{2021}{Veränderung 2015 bis 2021}
	\begin{spacing}{1} \scriptsize
		Anm.: Stand 2021\\
		Quelle: Eurostat (2024); Ber. \& Dar. imreg (2024) \end{spacing}
\end{figure}


\begin{figure}[p]
	\addcontentsline{toc}{subsection}{M+E-Beschäftigte je Betrieb}
	{\centering \maps{M+E-Beschäftigte je Betrieb}}
	\label{map:mebschdichte}
	\karte{ME_BeschProBetrieb}{2021}{Veränderung 2015 bis 2021}
	\begin{spacing}{1} \scriptsize
		Anm.: Stand 2021\\
		Quelle: Eurostat (2024); Ber. \& Dar. imreg (2024) \end{spacing}
\end{figure}


\begin{figure}[p]
	\addcontentsline{toc}{subsection}{M+E-Beschäftigte je Tsd. Einwohner}
	{\centering \maps{M+E-Beschäftigte je Tsd. Einwohner}}
	\label{map:mebschdichteew}
	\karte{ME_BeschProEW}{2021}{Veränderung 2015 bis 2021}
	\begin{spacing}{1} \scriptsize
		Anm.: Stand 2021\\
		Quelle: Eurostat (2024); Ber. \& Dar. imreg (2024) \end{spacing}
\end{figure}


\begin{figure}[p]
	\addcontentsline{toc}{subsection}{M+E-Beschäftigte je Tsd. Einwohner}
	{\centering \maps{Anteil M+E-Beschäftigter an allen Industriebeschäftigten}}
	\label{map:mebeschanteil}
	\karte{ME_AnteilBeschVG}{2020}{Veränderung 2015 bis 2020}
	\begin{spacing}{1} \scriptsize
		Anm.: Anteil an allen Beschäftigten im Verarbeitenden Gewerbe; Stand 2020\\
		Quelle: Eurostat (2024); Ber. \& Dar. imreg (2024) \end{spacing}
\end{figure}