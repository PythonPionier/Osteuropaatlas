%!TEX root = ../Osteuropaatlas.tex

\section{Bevölkerungsstruktur}

\begin{figure}[p]
	{\centering \maps{Einwohnerzahl}}
	\label{map:einwohner}
	\karte{Einwohnerzahl}{2022}{Veränderung 2015 bis 2022}
	\begin{spacing}{1} \scriptsize
		Anm.: Stand 2022\\
		Quelle: Eurostat (2024); Ber. \& Dar. imreg (2024) \end{spacing}
\end{figure}


\begin{figure}[p]
	{\centering \maps{Anzahl Haushalte}}
	\label{map:haushalte}
	\karte{Haushalte}{2022}{Veränderung 2015 bis 2022}
	\begin{spacing}{1} \scriptsize
		Anm.: Stand 2022\\
		Quelle: Eurostat (2024); Ber. \& Dar. imreg (2024) \end{spacing}
\end{figure}


\begin{figure}[p]
	{\centering \maps{Anzahl Haushalte in Städten}}
	\label{map:haushaltestadt}
	\karte{Haushalte_Stadt}{2022}{Veränderung 2015 bis 2022}
	\begin{spacing}{1} \scriptsize
		Anm.: Stand 2022\\
		Quelle: Eurostat (2024); Ber. \& Dar. imreg (2024) \end{spacing}
\end{figure}


\begin{figure}[p]
	{\centering \maps{Bevölkerungsdichte}}
	\label{map:bevdichte}
	\karte{Bevoelkerungsdichte}{2022}{Veränderung 2015 bis 2022}
	\begin{spacing}{1} \scriptsize
		Anm.: Stand 2022\\
		Quelle: Eurostat (2024); Ber. \& Dar. imreg (2024) \end{spacing}
\end{figure}


\begin{figure}[p]
	{\centering \maps{Bevölkerungsveränderung}}
	\label{map:bevrate}
	\karte{Bevoelkerungsveraenderung_Rate}{2021}{Veränderung 2015 bis 2021}
	\begin{spacing}{1} \scriptsize
		Anm.: Bevölkerungsveränderung je Tsd. Einwohner; Stand 2021\\
		Quelle: Eurostat (2024); Ber. \& Dar. imreg (2024) \end{spacing}
\end{figure}


\begin{figure}[p]
	{\centering \maps{Natürliche Bevölkerungsveränderung}}
	\label{map:natbevrate}
	\karte{NatBevoelkerungsveraenderung_Rate}{2021}{Veränderung 2015 bis 2021}
	\begin{spacing}{1} \scriptsize
		Anm.: Bevölkerungsveränderung je Tsd. Einwohner; Natürliche Bevölkerungsveränderung = Lebendgeburten $-$ Sterbefälle; Stand 2021\\
		Quelle: Eurostat (2024); Ber. \& Dar. imreg (2024) \end{spacing}
\end{figure}


\begin{figure}[p]
	{\centering \maps{Wanderungssaldo}}
	\label{map:wanderung}
	\karte{Wanderungssaldo_Rate}{2021}{Veränderung 2015 bis 2021}
	\begin{spacing}{1} \scriptsize
		Anm.: Wanderungssaldo je Tsd. Einwohner; Wanderungssaldo = Zuzüge $-$ Fortzüge; Stand 2021\\
		Quelle: Eurostat (2024); Ber. \& Dar. imreg (2024) \end{spacing}
\end{figure}


\begin{figure}[p]
	{\centering \maps{Altersdurchschnitt (Median)}}
	\label{map:alter}
	\karte{Medianalter}{2022}{Veränderung 2015 bis 2022}
	\begin{spacing}{1} \scriptsize
		Anm.: Stand 2022\\
		Quelle: Eurostat (2024); Ber. \& Dar. imreg (2024) \end{spacing}
\end{figure}


\begin{figure}[p]
	{\centering \maps{Frauenquotient}}
	\label{map:frauen}
	\karte{Frauenquotient}{2022}{Veränderung 2015 bis 2022}
	\begin{spacing}{1} \scriptsize
		Anm.: Frauenquotient = $\frac{\text{Anzahl Frauen}}{\text{Anzahl Männer}} \times 100$; Stand 2022\\
		Quelle: Eurostat (2024); Ber. \& Dar. imreg (2024) \end{spacing}
\end{figure}


\begin{figure}[p]
	{\centering \maps{Jugendquotient}}
	\label{map:jugend}
	\karte{Jugendquotient}{2022}{Veränderung 2015 bis 2022}
	\begin{spacing}{1} \scriptsize
		Anm.: Jugendquotient = $\frac{\text{Personen 0 - 14 Jahre}}{\text{Personen 15 - 64 Jahre}} \times 100$; Stand 2022\\
		Quelle: Eurostat (2024); Ber. \& Dar. imreg (2024) \end{spacing}
\end{figure}


\begin{figure}[p]
	{\centering \maps{Altenquotient}}
	\label{map:alte}
	\karte{Altenquotient}{2022}{Veränderung 2015 bis 2022}
	\begin{spacing}{1} \scriptsize
		Anm.: Altenquotient = $\frac{\text{Personen $\geq$ 65 Jahre}}{\text{Personen 15 - 64 Jahre}} \times 100$; Stand 2022\\
		Quelle: Eurostat (2024); Ber. \& Dar. imreg (2024) \end{spacing}
\end{figure}


\begin{figure}[p]
	{\centering \maps{Altersabhängigkeitsquotient}}
	\label{map:altersabh}
	\karte{Altersabhaengigkeitsquotient}{2022}{Veränderung 2015 bis 2022}
	\begin{spacing}{1} \scriptsize
		Anm.: Altersabhängigkeitsquotient = $\frac{\text{Personen 0 - 14 oder $\geq$ 65 Jahre}}{\text{Personen 15 - 64 Jahre}} \times 100$; Stand 2022\\
		Quelle: Eurostat (2024); Ber. \& Dar. imreg (2024) \end{spacing}
\end{figure}


