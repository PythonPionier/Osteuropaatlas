%!TEX root = ../Osteuropaatlas.tex

\section{Arbeitsmarkt}

\begin{figure}[p]
	{\centering \maps{Erwerbstätige}}
	\label{map:erwerb}
	\karte{Erwerbstaetige}{2022}{Veränderung 2015 bis 2022}
	\begin{spacing}{1} \scriptsize
		Anm.: Personen im Alter von 15 bis 64 Jahre; Stand 2022\\
		Quelle: Eurostat (2024); Ber. \& Dar. imreg (2024) \end{spacing}
\end{figure}


\begin{figure}[p]
	{\centering \maps{Erwerbstätigenquote}}
	\label{map:erwerbquote}
	\karte{Erwerbstaetigenquote}{2022}{Veränderung 2015 bis 2022}
	\begin{spacing}{1} \scriptsize
		Anm.: Anteil an Gesamtbevölkerung; Personen im Alter von 15 bis 64 Jahre; Stand 2022\\
		Quelle: Eurostat (2024); Ber. \& Dar. imreg (2024) \end{spacing}
\end{figure}


\begin{figure}[p]
	{\centering \maps{Teilzeitquote}}
	\label{map:teilzeit}
	\karte{Teilzeitquote}{2022}{Veränderung 2015 bis 2022}
	\begin{spacing}{1} \scriptsize
		Anm.: Anteil an allen Erwerbstätigen; Personen im Alter von 15 bis 64 Jahre; Stand 2022\\
		Quelle: Eurostat (2024); Ber. \& Dar. imreg (2024) \end{spacing}
\end{figure}


\begin{figure}[p]
	{\centering \maps{Erwerbstätigenquote Frauen}}
	\label{map:erwerbfrauen}
	\karte{Erwerbstaetigenquote_Frauen}{2022}{Veränderung 2015 bis 2022}
	\begin{spacing}{1} \scriptsize
		Anm.: Anteil an allen Frauen; Personen im Alter von 15 bis 64 Jahre; Stand 2022\\
		Quelle: Eurostat (2024); Ber. \& Dar. imreg (2024) \end{spacing}
\end{figure}


\begin{figure}[p]
	{\centering \maps{Erwerbstätigenquote Männer}}
	\label{map:erwerbmaenner}
	\karte{Erwerbstaetigenquote_Maenner}{2022}{Veränderung 2015 bis 2022}
	\begin{spacing}{1} \scriptsize
		Anm.: Anteil an allen Männern; Personen im Alter von 15 bis 64 Jahre; Stand 2022\\
		Quelle: Eurostat (2024); Ber. \& Dar. imreg (2024) \end{spacing}
\end{figure}


\begin{figure}[p]
	{\centering \maps{Erwerbstätigenquote Niedrigqualifizierter}}
	\label{map:erwerbniedrig}
	\karte{Erwerbstaetigenquote_Bildung02}{2022}{Veränderung 2015 bis 2022}
	\begin{spacing}{1} \scriptsize
		Anm.: Anteil an allen Niedrigqualifizierten; ISCED Stufen 0 bis 2; Personen im Alter von 15 bis 64 Jahre; Stand 2022\\
		Quelle: Eurostat (2024); Ber. \& Dar. imreg (2024) \end{spacing}
\end{figure}


\begin{figure}[p]
	{\centering \maps{Erwerbstätigenquote Mittelqualifizierter}}
	\label{map:erwerbmittel}
	\karte{Erwerbstaetigenquote_Bildung34}{2022}{Veränderung 2015 bis 2022}
	\begin{spacing}{1} \scriptsize
		Anm.: Anteil an allen Mittelqualifizierten; ISCED Stufen 3 bis 4; Personen im Alter von 15 bis 64 Jahre; Stand 2022\\
		Quelle: Eurostat (2024); Ber. \& Dar. imreg (2024) \end{spacing}
\end{figure}

\begin{figure}[p]
	{\centering \maps{Erwerbstätigenquote Hochqualifizierter}}
	\label{map:erwerbhoch}
	\karte{Erwerbstaetigenquote_Bildung58}{2022}{Veränderung 2015 bis 2022}
	\begin{spacing}{1} \scriptsize
		Anm.: Anteil an allen Hochqualifizierten; ISCED Stufen 5 bis 8; Personen im Alter von 15 bis 64 Jahre; Stand 2022\\
		Quelle: Eurostat (2024); Ber. \& Dar. imreg (2024) \end{spacing}
\end{figure}


\begin{figure}[p]
	{\centering \maps{Erwerbstätigenquote im Inland Geborener}}
	\label{map:erwerbinland}
	\karte{Erwerbstaetigenquote_Inlaender}{2022}{Veränderung 2015 bis 2022}
	\begin{spacing}{1} \scriptsize
		Anm.: Anteil an im Inland Geborenen; Geburtsland = Land der Erwerbstätigkeit; Personen im Alter von 15 bis 64 Jahre; Stand 2022\\
		Quelle: Eurostat (2024); Ber. \& Dar. imreg (2024) \end{spacing}
\end{figure}


\begin{figure}[p]
	{\centering \maps{Erwerbstätigenquote im Ausland Geborener}}
	\label{map:erwerbausland}
	\karte{Erwerbstaetigenquote_Auslaender}{2022}{Veränderung 2017 bis 2022}
	\begin{spacing}{1} \scriptsize
		Anm.: Anteil an im Ausland Geborenen; Geburtsland $\neq$ Land der Erwerbstätigkeit; Personen im Alter von 15 bis 64 Jahre; Stand 2022\\
		Quelle: Eurostat (2024); Ber. \& Dar. imreg (2024) \end{spacing}
\end{figure}


\begin{figure}[p]
	{\centering \maps{Jugenderwerbstätigenquote}}
	\label{map:erwerbjugend}
	\karte{Jugenderwerbstaetigenquote}{2022}{Veränderung 2015 bis 2022}
	\begin{spacing}{1} \scriptsize
		Anm.: Anteil an allen Jugendlichen; Personen im Alter von 15 bis 24 Jahre; Stand 2022\\
		Quelle: Eurostat (2024); Ber. \& Dar. imreg (2024) \end{spacing}
\end{figure}


\begin{figure}[p]
	{\centering \maps{Erwerbspersonen}}
	\label{map:erwerbspers}
	\karte{Erwerbspersonen}{2022}{Veränderung 2015 bis 2022}
	\begin{spacing}{1} \scriptsize
		Anm.: Erwerbspersonen = Berufstätige \& Arbeitslose; Personen ab 15 Jahre; Stand 2022\\
		Quelle: Eurostat (2024); Ber. \& Dar. imreg (2024) \end{spacing}
\end{figure}


\begin{figure}[p]
	{\centering \maps{Weibliche Erwerbspersonen}}
	\label{map:erwerbspersfrauen}
	\karte{Erwerbspersonen_Frauen}{2022}{Veränderung 2015 bis 2022}
	\begin{spacing}{1} \scriptsize
		Anm.: Erwerbspersonen = Berufstätige \& Arbeitslose; Personen ab 15 Jahre; Stand 2022\\
		Quelle: Eurostat (2024); Ber. \& Dar. imreg (2024) \end{spacing}
\end{figure}


\begin{figure}[p]
	{\centering \maps{Männliche Erwerbspersonen}}
	\label{map:erwerbspersmaenner}
	\karte{Erwerbspersonen_Maenner}{2022}{Veränderung 2015 bis 2022}
	\begin{spacing}{1} \scriptsize
		Anm.: Erwerbspersonen = Berufstätige \& Arbeitslose; Personen ab 15 Jahre; Stand 2022\\
		Quelle: Eurostat (2024); Ber. \& Dar. imreg (2024) \end{spacing}
\end{figure}


\begin{figure}[p]
	{\centering \maps{Niedrigqualizifierte Erwerbspersonen}}
	\label{map:erwerbspersniedrig}
	\karte{Erwerbspersonen_Bildung02}{2022}{Veränderung 2015 bis 2022}
	\begin{spacing}{1} \scriptsize
		Anm.: Erwerbspersonen = Berufstätige \& Arbeitslose; ISCED Stufen 0 bis 2; Personen ab 15 Jahre; Stand 2022\\
		Quelle: Eurostat (2024); Ber. \& Dar. imreg (2024) \end{spacing}
\end{figure}


\begin{figure}[p]
	{\centering \maps{Mittelqualifizierte Erwerbspersonen}}
	\label{map:erwerbspersmittel}
	\karte{Erwerbspersonen_Bildung34}{2022}{Veränderung 2015 bis 2022}
	\begin{spacing}{1} \scriptsize
		Anm.: Erwerbspersonen = Berufstätige \& Arbeitslose; ISCED Stufen 3 bis 4; Personen ab 15 Jahre; Stand 2022\\
		Quelle: Eurostat (2024); Ber. \& Dar. imreg (2024) \end{spacing}
\end{figure}


\begin{figure}[p]
	{\centering \maps{Hochqualifizierte Erwerbspersonen}}
	\label{map:erwerbspershoch}
	\karte{Erwerbspersonen_Bildung58}{2022}{Veränderung 2015 bis 2022}
	\begin{spacing}{1} \scriptsize
		Anm.: Erwerbspersonen = Berufstätige \& Arbeitslose; ISCED Stufen 5 bis 8; Personen ab 15 Jahre; Stand 2022\\
		Quelle: Eurostat (2024); Ber. \& Dar. imreg (2024) \end{spacing}
\end{figure}


\begin{figure}[p]
	{\centering \maps{Arbeitslosenquote}}
	\label{map:arbeitslosenquote}
	\karte{Arbeitslosenquote}{2022}{Veränderung 2015 bis 2022}
	\begin{spacing}{1} \scriptsize
		Anm.: Personen im Alter von 15 bis 74 Jahren; Stand 2022\\
		Quelle: Eurostat (2024); Ber. \& Dar. imreg (2024) \end{spacing}
\end{figure}


\begin{figure}[p]
	{\centering \maps{Arbeitslosenquote Frauen}}
	\label{map:arbeitslosenquotefrauen}
	\karte{Arbeitslosenquote_Frauen}{2022}{Veränderung 2015 bis 2022}
	\begin{spacing}{1} \scriptsize
		Anm.: Personen im Alter von 15 bis 74 Jahren; Stand 2022\\
		Quelle: Eurostat (2024); Ber. \& Dar. imreg (2024) \end{spacing}
\end{figure}


\begin{figure}[p]
	{\centering \maps{Arbeitslosenquote Männer}}
	\label{map:arbeitslosenquotemaenner}
	\karte{Arbeitslosenquote_Maenner}{2022}{Veränderung 2015 bis 2022}
	\begin{spacing}{1} \scriptsize
		Anm.: Personen im Alter von 15 bis 74 Jahren; Stand 2022\\
		Quelle: Eurostat (2024); Ber. \& Dar. imreg (2024) \end{spacing}
\end{figure}


\begin{figure}[p]
	{\centering \maps{Arbeitslosenquote Niedrigqualifizierter}}
	\label{map:arbeitslosenquoteniedrig}
	\karte{Arbeitslosenquote_Bildung02}{2022}{Veränderung 2015 bis 2022}
	\begin{spacing}{1} \scriptsize
		Anm.: ISCED Stufen 0 bis 2; Personen im Alter von 15 bis 74 Jahren; Stand 2022\\
		Quelle: Eurostat (2024); Ber. \& Dar. imreg (2024) \end{spacing}
\end{figure}


\begin{figure}[p]
	{\centering \maps{Arbeitslosenquote Mittelqualifizierter}}
	\label{map:arbeitslosenquotemittel}
	\karte{Arbeitslosenquote_Bildung34}{2022}{Veränderung 2015 bis 2022}
	\begin{spacing}{1} \scriptsize
		Anm.: ISCED Stufen 3 bis 4; Personen im Alter von 15 bis 74 Jahren; Stand 2022\\
		Quelle: Eurostat (2024); Ber. \& Dar. imreg (2024) \end{spacing}
\end{figure}


\begin{figure}[p]
	{\centering \maps{Arbeitslosenquote Hochqualifizierter}}
	\label{map:arbeitslosenquotehoch}
	\karte{Arbeitslosenquote_Bildung58}{2022}{Veränderung 2015 bis 2022}
	\begin{spacing}{1} \scriptsize
		Anm.: ISCED Stufen 5 bis 8; Personen im Alter von 15 bis 74 Jahren; Stand 2022\\
		Quelle: Eurostat (2024); Ber. \& Dar. imreg (2024) \end{spacing}
\end{figure}


\begin{figure}[p]
	{\centering \maps{Ungenutzes Arbeitsmarktpotential}}
	\label{map:potential}
	\karte{Arbeitsmarktpotential}{2022}{Veränderung 2015 bis 2022}
	\begin{spacing}{1} \scriptsize
		Anm.: Anteil an der erweiterten Erwerbsbevölkerung; Personen im Alter von mind. 15 Jahren; Stand 2022\\
		Quelle: Eurostat (2024); Ber. \& Dar. imreg (2024) \end{spacing}
\end{figure}


\begin{figure}[p]
	{\centering \maps{Ungenutzes Arbeitsmarktpotential Frauen}}
	\label{map:potentialfrauen}
	\karte{Arbeitsmarktpotential_Frauen}{2022}{Veränderung 2015 bis 2022}
	\begin{spacing}{1} \scriptsize
		Anm.: Anteil an der erweiterten weiblichen Erwerbsbevölkerung; Personen im Alter von mind. 15 Jahren; Stand 2022\\
		Quelle: Eurostat (2024); Ber. \& Dar. imreg (2024) \end{spacing}
\end{figure}


\begin{figure}[p]
	{\centering \maps{Ungenutzes Arbeitsmarktpotential Männer}}
	\label{map:potentialmaenner}
	\karte{Arbeitsmarktpotential_Maenner}{2022}{Veränderung 2015 bis 2022}
	\begin{spacing}{1} \scriptsize
		Anm.: Anteil an der erweiterten männlichen Erwerbsbevölkerung; Personen im Alter von mind. 15 Jahren; Stand 2022\\
		Quelle: Eurostat (2024); Ber. \& Dar. imreg (2024) \end{spacing}
\end{figure}


\begin{figure}[p]
	{\centering \maps{Arbeitsmarktnachfrage IKT}}
	\label{map:ikt}
	\karte{Arbeitsmarktnachfrage_IKT}{2023}{Veränderung 2019-Q4 bis 2023-Q3}
	\begin{spacing}{1} \scriptsize
		Anm.: Anteil an Online-Stellenanzeigen; Stand 2023\\
		Quelle: Eurostat (2024); Ber. \& Dar. imreg (2024) \end{spacing}
\end{figure}


\begin{figure}[p]
	{\centering \maps{Wochenarbeitszeit}}
	\label{map:zeit}
	\karte{Arbeitszeit}{2022}{Veränderung 2015 bis 2022}
	\begin{spacing}{1} \scriptsize
		Anm.: Vertragliche Wochenarbeitszeit; Personen im Alter von 15 bis 64 Jahre; Stand 2022\\
		Quelle: Eurostat (2024); Ber. \& Dar. imreg (2024) \end{spacing}
\end{figure}


\begin{figure}[p]
	{\centering \maps{Langzeitbeschäftigung}}
	\label{map:lang}
	\karte{Langzeitbeschaeftigung}{2022}{Veränderung 2015 bis 2022}
	\begin{spacing}{1} \scriptsize
		Anm.: Anteil Beschäftigungsverhältnisse mit mind. fünf Jahren Vertragsbestehen/ -dauer; Personen im Alter von 15 bis 64 Jahre; Stand 2022\\
		Quelle: Eurostat (2024); Ber. \& Dar. imreg (2024) \end{spacing}
\end{figure}


\begin{figure}[p]
	{\centering \maps{Geschlechtsspezifischer Beschäftigungsunterschied}}
	\label{map:geschlecht}
	\karte{Geschlechterbeschaeftigung}{2022}{Veränderung 2015 bis 2022}
	\begin{spacing}{1} \scriptsize
		Anm.: Unterschied Erwerbstätigenquote Männer zu Frauen im Alter von 20 bis 64 Jahre; Stand 2022\\
		Quelle: Eurostat (2024); Ber. \& Dar. imreg (2024) \end{spacing}
\end{figure}
